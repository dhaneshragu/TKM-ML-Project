% To learn more - Search `beamer-template` or see this https://www.overleaf.com/learn/latex/Beamer

\documentclass{beamer}
\usepackage[utf8]{inputenc}

\usetheme{Madrid}
\usecolortheme{default}

%------------------------------------------------------------
%This block of code defines the information to appear in the
%Title page
\title[Threshold Based \textit{K}-Means Monitoring] %optional
{Continuous k-Means Monitoring
over Moving Objects}

\subtitle{A Threshold Based Approach}

\author[Group 7] % (optional)
{Aasneh Prasad \and Gautam Sharma \and Pranav Jain \and Sahil Danayak \and Sreehari \and Dhanesh}


\date[16th November 2023] % (optional)
{\textbf{DA321M Course Project, July-Nov 2023}}

% \logo{\includegraphics[height=0.6cm]{iitg-logo.jpeg}}

%End of title page configuration block
%------------------------------------------------------------



%------------------------------------------------------------
%The next block of commands puts the table of contents at the 
%beginning of each section and highlights the current section:

\AtBeginSection[]
{
  \begin{frame}
    \frametitle{Table of Contents}
    \tableofcontents[currentsection]
  \end{frame}
}
%------------------------------------------------------------


\begin{document}

%The next statement creates the title page.
\frame{\titlepage}


%---------------------------------------------------------
%This block of code is for the table of contents after
%the title page
\begin{frame}
  \frametitle{Table of Contents}
  \tableofcontents
\end{frame}
%---------------------------------------------------------


%---------------------------------------------------------
\section{Introduction}
\begin{frame}
  \frametitle{Introduction}

  \vspace{-1mm}
  Notations used in legacy KNN algorithms :-
  \begin{itemize}
      \item Data set \(P=\{p_1, \ldots, p_n\}\). Query returns centers \(M=\{m_1, \ldots, m_k\}\).
      \item \(cost(M) = \left( \sum_{i=1}^{n} dist^2(p_i, NN(p_i, M)) \right)\) for \(i\) from 1,2,\ldots,\(n\)
      % \item \(NN(p_i, M)\) is the nearest neighbor of \(p_i\) in \(M\).
      % \item \(\theta_i\) is threshold of object \(p_i\). The object issues an update only when its current location deviates by \(\theta_i.\)
  \end{itemize}
  \vspace{2mm}

    \textbf{Reference Solution:} Whenever an object moves, it sends a location update. \\

    \textbf{Threshold based:} Issues an update only if its current location deviates from the previous one by at least $\delta_i$.

  \vspace{2mm}

  ZDT-Algorithm :-
  \begin{itemize}
      \item Let \(M\) be the current center set in HC algorithm such that no center can deviate from its position in \(M\) by more than \(\delta\).
      \item HC will converge \(M\) to \(M\)* where \(cost(M^*) \geq cost(M)-n\delta^2 \)
      % \item Zhang et al.[1] provide a method to compute \(\delta\) in \(O(nlogn)\) time 
      \item If this bound exceeds the best cost achieved by a previous local optimum (with a different seed set), subsequent iterations of HC can be pruned
  \end{itemize}
\end{frame}
%---------------------------------------------------------


%---------------------------------------------------------
\section{HC* Algorithm}
\begin{frame}
  \frametitle{HC* Algorithm}

  \begin{columns}

    \column{0.45\textwidth}
    \begin{itemize}
      \item Uses data from the previous iterations to form new clusters.
      \item  It exploits the fact that only points close to the perpendicular bisectors defined by the centers, may change clusters.
      \item Less computationally expensive
      \item $P_{active}$ of points whose clusters have been reassigned.

    \end{itemize}

    \column{0.5\textwidth}
    \includegraphics[width=1\textwidth]{Pic1.png}

  \end{columns}
\end{frame}

%---------------------------------------------------------
\section{Assignment of thresholds}
\begin{frame}
  \frametitle{Assignment of thresholds}
  The threshold assignment routine takes as input the
  objects locations \(P = \{p1, p2, \ldots, pn\}\) and the k-means set
  \(\{m1, m2, \ldots, mk\}\), and outputs a set of \(n\) real values \(\{\theta 1, \theta 2, \ldots, \theta n\}\), i.e., the thresholds.
  \begin{itemize}
    \item It can be derived that \[ \theta_p \leq \min_{m \in M \setminus \{m_p\}} \text{dist}(p, m) - \triangle \], where $\triangle$ represents the maximum permissible error.
          \vspace{5mm}
    \item Based on the following observation we can develop an algorithm as given below for calculating the thresholds or maximum distance each object can move without sending a signal to the server in its neighbourhood
  \end{itemize}


\end{frame}


\begin{frame}
  \frametitle{Algorithm for calculating thresholds}


  \begin{center}
    \includegraphics[width=\textwidth]{algo.png} % Replace with the actual file name

  \end{center}
\end{frame}
%------------------------------------------------------------

%---------------------------------------------------------
\section{Utilizing Object Speed and Dissemination of Thresholds}
\begin{frame}
  \frametitle{Utilizing Object Speed and Dissemination of Thresholds}
  Assume the system knows roughly the average speed $s_{i}$ of each object $p_{i}$, then the objective function becomes $\sum\limits_{i=1}^n s_i/\theta_i$.
  \begin{itemize}
    \item Optimal solution using Lagrange multiplier: $\forall i,\space \theta_{i} = (n\triangle_{1} \sqrt{s_{i}})/(\sum\limits_{j=1}^n \sqrt{s_{j}})$.
    \item Methods of Dissemination of Thresholds (depending on the computational capabilities of the objects) :
          \begin{itemize}
            \item Broadcasting method: Initially, the
                  server broadcasts $\triangle$. Then each object
                  computes its own threshold based on the broadcast
                  information.
            \item If objects have limited
                  computational capabilities then initially, the server sends $\triangle_{1}$ to
                  all objects through single-cast messages. In subsequent
                  updates, it sends the threshold to an object only when it has
                  changed.
          \end{itemize}
  \end{itemize}


\end{frame}



%---------------------------------------------------------
\section{Comparision of CPU Times}
\begin{frame}
  \frametitle{Comparision of CPU Times}

  \begin{columns}

    \column{0.35\textwidth}
    \begin{itemize}
      \item This graph shows the comparison of CPU times of \texttt{REF} (Normal K-means algorithm) v/s \texttt{TKM} (Threshold-based K-Means algorithm)
      \item At each iteration, some new points were added and, with a probability, the previous points were displaced by a small amount.


    \end{itemize}

    \column{0.65\textwidth}
    \includegraphics[width=1\textwidth]{Diff.png}
    \begin{itemize}
      \vspace{-0.2cm}
      \item \small This graph shows \texttt{TKM} performs much better in most of the cases.
    \end{itemize}
  \end{columns}
\end{frame}

\section{Comparision of Total Messages sent}
\begin{frame}
  \frametitle{Comparision of Total Messages sent}

  \begin{columns}

    \column{0.35\textwidth}
    \begin{itemize}
      \item This graph shows the comparison of messages sent to the server for \texttt{REF} (Normal K-means algorithm) v/s \texttt{TKM} (Threshold-based K-Means algorithm)
      \item At each iteration, some new points were added and, with a probability, the previous points were displaced by a small amount.


    \end{itemize}

    \column{0.65\textwidth}
    \includegraphics[width=1\textwidth]{msg.png}
    \begin{itemize}
      \vspace{-0.2cm}
      \item \small This graph shows \texttt{TKM} sends less messages to the server, ensuring faster downtime for the same results.
    \end{itemize}
  \end{columns}
\end{frame}
%---------------------------------------------------------



\end{document}
